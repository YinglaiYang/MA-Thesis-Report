% Yinglai Yang
%
% Notes about the compressed sensing model for the multicarrier MIMO system
%

\documentclass[12pt, a4paper]{scrreprt}

\usepackage[utf8]{inputenc}

\usepackage{amsmath}
\usepackage{amsfonts}

\usepackage[T1]{fontenc}

\usepackage{graphicx}
\usepackage[format=hang]{caption}       % for hanging captions
\usepackage{subfig}                     % for subfigures
\usepackage{wrapfig}                    % for figures floating in text,

% declarations
\newcommand{\matlab}{\textsc{Matlab}\raisebox{1ex}{\tiny{\textregistered}} }
% Integers, natural, real and complex numbers
\newcommand{\Z}{\mathbb{Z}}
\newcommand{\N}{\mathbb{N}}
\newcommand{\R}{\mathbb{R}}
\newcommand{\C}{\mathbb{C}}
% expectation operator
\newcommand{\E}{\operatorname{E}}
% imaginary unit
\newcommand{\im}{\operatorname{j}}
% Euler's number with exponent as parameter, e.g. \e{\im\omega}
\newcommand{\e}[1]{\operatorname{e}^{\,#1}}
% short command for \operatorname{}
\newcommand{\op}[1]{\operatorname{#1}}

\newcommand{\mat}[1]{\mathbf{#1}}


% ===========================================================================
\begin{document}
  \chapter{Relation to Compressed Sensing}
  \label{chap:Relation to Compressed Sensing}

  \section{Overview of Compressed Sensing}
  \label{sec:Overview of Compressed Sensing}

  \subsection{Compressed Sensing Formulation}
  \label{sub:subsection label}
  \begin{equation}
    \vec{x} = \mat{\Phi} \vec{y}
    \label{eq:CS:Dense to compressed}
  \end{equation}

  \begin{equation}
    \vec{x} = \mat{\Phi} \mat{\Psi} \vec{\theta}

  \end{equation}

  \subsection{Conditions for successful Parameter Estimation}
  \label{sub:Conditions for successful Parameter Estimation}


  \section{Coherence and Sidelobe Level}
  \label{sec:Coherence and Sidelobe Level}

  \section{The Compressed Sensing Model for MC-MIMO}
  \label{sec:The Compressed Sensing Model for MC-MIMO}
  The sparse MC-MIMO antenna array can be related to compressed sensing. The received signal from the sparse array represents the compressed signal $\vec{x}$, while the dense signal $\vec{y}$ from eq. \ref{eq:CS:Dense to compressed} is equivalent to the signal retrieved from a ``dense'' grid of available antenna positions.

  Eq. \ref{eq:CS:Dense to compressed} also describes the relation between this grid of antenna positions, in which we distribute a small number of antennas to arrive at the sparse antenna geometry. The selection matrix $\mat{\Phi}$ is simply a matrix compiled of the rows of an identity matrix. The rows that are selected equate to the potential antenna positions where an actual antenna is placed.

  \subsection{The Sparse Parameter Space}
  \label{sub:The Sparse Parameter Space}

  \subsection{Vectorizing Signal and Parameter Space}
  \label{sub:Vectorizing Signal and Parameter Space}

  \subsection{The Sensing Matrix}
  \label{sub:The Sensing Matrix}

  \section{Analysis of the Compressed Sensing Model for MC-MIMO}
  \label{sec:Analysis of the Compressed Sensing Model for MC-MIMO}

\end{document}
